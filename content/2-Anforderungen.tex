\section{Anforderungen}

	\subsection{Nutzungsbeispiel}
    \newpage
    
	\subsection{Eingabe}
    	Dank der Vielzahl an möglichen Behinderungen und Einschränkungen gibt es eine mindestens genau so große Zahl an Eingabemethoden und Geräten. In vielen Fällen sind Eingabegerät und Software eng miteinander verbunden oder sogar ein in sich abgeschlossenes System.
        
        Um möglichst viele verschiedene Eingabegeräte zu unterstützen soll die zu entwickelnde Applikation nicht direkt auf spezielle Geräte angepasst werden. Darum werden hier nicht bestimmte Aktionen zur Eingabe beschrieben sondern "Signale" welche dann von verschieden Geräten erzeugt werden können.
        
        Für die Entwicklung des Prototypen können Signale so auch sehr einfach über Tastatureingaben simuliert werden. Es ist auch denkbar komplizierte Eingabemethoden zu simulieren um auch die tatsächliche Nutzerfreundlichkeit besser darzusetllen. Schließlich ist das Tippen einer Taste doch sehr viel einfacher als z.B. das Betätigen eines Kopfschalters an einem Rollstuhl. 
        
        Da nicht jedes Eingabegerät die gleiche Anzahl an Signalen liefert soll hier ein Minimum von drei diskreten Signalen festgelegt werden. Mit diesen drei Signalen sollen alle Funktionen der Applikation umzusetzen sein. Dabei ist es möglich, dass das gleiche Signal in einem anderen Kontext eine andere Bedeutung hat. Es ist auch vorstellbar, wenn unbedingt nötig, eine weitere Funktion durch Wiederholen des gleichen Signals innerhalb eines gewissen Zeitraums aufzurufen.
        
        Diese Einschränkung wäre natürlich für Nutzer von Eingabegeräten welche mehr als drei Signale erzeugen können schnell frustrierend. So soll es möglich sein auf weitere Signale zu reagieren und mit diesen Erleichterungen in der Bedienung der Applikation umzusetzen. Ein Beispiel hierfür wäre eine Liste, die sich über mehrere Zeilen erstreckt. Es ist möglich durch diese mit einem einzigen "weiter" Befehl zu navigieren. Wenn möglich wäre es aber auch wünschenswert eine Zeile nach unten oder eben auch zurück zu navigieren.
        
        Bei der Betrachtung bestehender Eingabegeräte fallen weitere Anforderungen an die Bedienung der Applikation auf:
        
        So gibt es verschiedene Eingabegeräte die zur Steuerung eines Mauszeigers gedacht sind. Meist scheint hier die Bedienung der Maus etwas komplizierter als mit einer klassischen Computermaus. Auch gibt es die verschiedensten Bauweisen von meist großen Tastern oder direkt Touchscreens. Darum sollte es auch möglich sein die Schaltflächen der Applikation klassisch per Maus oder Fingerdruck auswählen zu können. Dabei sollte auf die ungewöhnliche Steuerung des Mauszeigers oder motorische Einschränkungen bei der Bedienung von Touchscreens Rücksicht genommen werden. 
        
        Ziel dieser Art von Eingabehandhabung ist zum einen die Konzentration der Arbeit und des Prototypen auf die Sprachanalyse und die visualisierung der Ergebnisse. Zum Anderen aber auch der Versuch die Grundlagen für ein Modulares System zu skizzieren welches auch mit zukünftigen noch nicht existierenden Eingabegeräten funktionieren kann. Ein Beispiel hierfür wäre der bereits erwähnte Mauszeiger der eben nicht nur über eine klassische Maus gesteuert werden kann.
        \newpage
	\subsection{Kathegorie-auswahl / -erkennung}
	\subsection{Prediktion}
	\subsection{Ausgabe}