\subsection{Corpora}
    Die für \emph{Statistische Sprachmodelle} verwendeten Trainingstexte werden auch als \emph{Textcorpus} bezeichnet. Wie bereits in \autoref{sec:langmodels} erwähnt kann eine Thematische ausrichtung des Corpus ein Sprachmodell stark beinflussen. So können aus unterschiedlichen Themenfeldern bezogene Texte nicht nur unterschiedliche Häufigkeiten, sondern auch ein unterschiedliches Vokabular haben.
        
	Es kann nun also versucht werden einen \emph{Corpus} zu bilden der eine Sprache möglichst umfassend und in möglichst vielen Themenbereichen abbildet. Die Penn Treebank wäre ein Beispiel für einen solchen \emph{Corpus} für die Englische Sprache. Ein anderer Ansatz wäre das Sprachmodell auf ein bestimmtes Themengebiet zu speialisieren (wie claßen paper mit medizin????). Es ist zu erwarten, dass ein solches Modell zwar im Allgemeinen schlechtere Vorhersagen treffen kann, dafür aber in seinem Themengebiet besser als ein allgemeines Modell abschneidet.
        
        Um dem in \autoref{sec:n-gramms} erwähnten Problem der \emph{data sparsity} vorzubeugen muss ein \emph{Corpus} ausreichend groß sein. Brown u. a. nutzen für ihre versuche \emph{Corpusgrößen} von xxx xxx und sogar xxx (cite). Die Penn Treebank umfasst xxxx Worte (cite). Umso kleiner ein Corpus ist, desto stärker wird er nur den gelernten Text beschreiben und nicht eine Sprache im allgemeinen.